\section{MDP}

A \textcolor{red}{reward} $R_t$ is a scalar feedback signal: indicates how well agent is doing at step $t$.

\textcolor{orange}{The agent’s job is to select \underline{actions} to maximize (expected) cumulative reward!!}
\begin{itemize}
	\item Actions may have long term consequences;
	\item Rewards may be delayed;
	\item It may be better to sacrifice immediate reward to gain more long-term reward.
\end{itemize}

The \textcolor{red}{agent state} $S^a_t$ is the agent's internal representation, i.e. whatever information it uses to pick the next action. Formally, state is a function of the history $S^a_t = f(H_t)$, where $H_t = O_1, R_1, A_1, ..., A_{t-1}, O_{t}, R_{t}$, i.e. all observable variables up to time $t$.

An \textcolor{red}{agent state} (a.k.a. \textcolor{red}{Markov state}) contains all useful information from the history. $S_t$ is Markov iff 
\begin{align*}
\Prob [S_{t+1}|S_t]	= \Prob[S_{t+1}| S_t, S_{t-1}, ... S_1]
\end{align*}
\begin{itemize}
	\item The future is independent of the past, given present.
	\item once the state is known the history can be thrown away.
	\item The state is sufficient statistics of the history.
\end{itemize}


(???) POMDP (partially observable) vs MDP


\boxedtext{
An \textcolor{red}{RL agent} may include one or more of these components:
\begin{itemize}
	\item \red{Policy}: agent’s behavior function
	\item \red{Value function}: how good is each state and/or action
	\item \red{Model}: agent’s representation of the environment
\end{itemize}
}

